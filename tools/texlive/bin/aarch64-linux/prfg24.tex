\documentclass[fleqn]{article}
\usepackage[left=1in, right=1in, top=1in, bottom=1in]{geometry}
\usepackage{mathexam}
\usepackage{amsmath}

\ExamClass{HTM26}
\ExamName{Pr\"ufung  aus Anwendung h\"oherer Mathematik}
\ExamHead{18. Dezember 2024}

\let\ds\displaystyle

\begin{document}
\begin{enumerate}
  \item L\"osen Sie folgende Differenzialgleichung mit der Bedingung $N(0)=N_0$ und bestimmen Sie den Typ und die Ordnung. Bei jedem radioaktiven Pr\"aparat ist die momentane \"Anderungsrate $\dot{N}$ des Bestandes entgegengesetzt proportional zum Bestand N selbst:
    \begin{equation}
      \dot{N}=-\lambda\cdot N
    \end{equation}
  Den Proportionalit\"atsfaktor $\lambda$ bezeichnet man als Zerfallskonstante.
    \item 
\end{enumerate}
\end{document}


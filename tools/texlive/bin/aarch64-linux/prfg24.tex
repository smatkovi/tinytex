\documentclass[fleqn]{article}
\usepackage[left=1in, right=1in, top=1in, bottom=1in]{geometry}
\usepackage{mathexam}
\usepackage{amsmath}

\ExamClass{HTM26}
\ExamName{Pr\"ufung  aus Anwendung h\"oherer Mathematik}
\ExamHead{18. Dezember 2024}

\let\ds\displaystyle

\begin{document}
Vorname:\\
Nachname:\\
Matrikelnummer:\\
\begin{enumerate}
  \item L\"osen Sie folgende Differenzialgleichung mit der Bedingung $N(0)=N_0$ und bestimmen Sie den Typ und die Ordnung. Bei jedem radioaktiven Pr\"aparat ist die momentane \"Anderungsrate $\dot{N}$ des Bestandes entgegengesetzt proportional zum Bestand N selbst:
    \begin{equation}
      \dot{N}=-\lambda\cdot N
    \end{equation}
  Den Proportionalit\"atsfaktor $\lambda$ bezeichnet man als Zerfallskonstante.
  \\8 Punkte
\item L\"osen Sie die Differenzialgleichung eines unged\"ampften LC-Serienschwingkreises, wobei eine Spule mit Induktivit\"at L und ein Kondensator mit Kapazit\"at C in Serie geschaltet werden.
  \begin{equation}
    \ddot{Q}+\frac{1}{L\cdot C}\cdot Q = 0
  \end{equation}
  Q ist hier die Ladung und $\dot{Q}$ w\"are theoretisch die Stromst\"arke I, die wir hier aber nicht verwenden.
 \\ 8 Punkte
\item L\"osen Sie die Differenzialgleichung des ged\"ampften, (periodisch) getriebenen, harmonischen Oszillators (Schwingung einer Feder, in mehreren Dimensionen eine Schaukel, oder wenn in einem Kristallgitter die Bindungen als Federn beschrieben werden, dann kommt pro zus\"atzlicher Bindung ein Term mit Federkonstante dazu)
  \begin{equation}
    \ddot{x}+2\cdot\gamma\cdot\dot{x}+\omega_0^2\cdot x = K\cdot \cos(\omega\cdot t)
  \end{equation}
    Dabei ist $\gamma$ die D\"ampfungskonstante, $\omega_0^2=\frac{D}{m}$, D die Federkonstante, die manchmal auch mit k bezeichnet wird und m die Masse.
  \\8 Punkte
\item Entwickeln Sie die Funktion $g(x)=-x$ f\"ur $x \in [-\pi;\pi)$ und sonst periodisch fortgesetzt, in eine Fourierreihe.
  \\8 Punkte
\item Bonusbeispiel: Separieren Sie die W\"armeleitungsgleichung und l\"osen Sie diese in einer Dimension. Welchem Typ geh\"ort sie an und von welcher Ordnung ist sie?
  \begin{equation}
    \dot{u} = a\cdot u''
  \end{equation}
  \\8 Bonuspunkte
\end{enumerate}
29-32 Punkte: Sehr gut, 25-28 Punkte: Gut, 20-24 Punkte: Befriedigend, 16-19 Punkte: Gen\"ugend
\end{document}


